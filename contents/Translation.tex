\cleardoublepage
\section{外文翻译}
\renewcommand{\theequation}{\arabic{equation}}
\renewcommand{\thefigure}{\arabic{figure}}
\begin{center}
    \zihao{3}\textbf{论文题目}
\end{center}
\subsection*{摘要}
这是摘要。
\settocdepth{part} % 以下部分不计入目录


% 注:正文部分的subsection中公式和图表的编号会在每一个subsection中重新开始计数,如果需要在整个文档中保持编号的连续性,可以在每个subsection后加上以下两行代码:
% \setcounter{equation}{上一个subsection中最后一个公式的编号}
% \setcounter{figure}{上一个subsection中最后一个图表的编号}

\subsection{引言}
这是引言,测试参考文献\cite{schweizer2013comparative}。

% 参考文献
\newpage
\bibliographystyle{gbt7714-numerical}
\phantomsection		% 要想目录中参考文献的超链接正确需要加这一语句
\subsection{参考文献}
{\normalfont\CJKfamily{Songti}\zihao{5}\setlength{\baselineskip}{14pt}
\renewcommand{\refname}{\vspace{-\baselineskip}}
\bibliography{reference/refs}}

\resettocdepth % 恢复目录计数